\documentclass[a4paper]{article} 
\addtolength{\hoffset}{-2.25cm}
\addtolength{\textwidth}{4.5cm}
\addtolength{\voffset}{-3.25cm}
\addtolength{\textheight}{5cm}
\setlength{\parskip}{0pt}
\setlength{\parindent}{0in}

%----------------------------------------------------------------------------------------
%	PACKAGES AND OTHER DOCUMENT CONFIGURATIONS
%----------------------------------------------------------------------------------------

\usepackage{blindtext} % Package to generate dummy text
\usepackage{charter} % Use the Charter font
\usepackage[utf8]{inputenc} % Use UTF-8 encoding
\usepackage{microtype} % Slightly tweak font spacing for aesthetics
\usepackage[english, ngerman]{babel} % Language hyphenation and typographical rules
\usepackage{amsthm, amsmath, amssymb} % Mathematical typesetting
\usepackage{float} % Improved interface for floating objects
\usepackage[final, colorlinks = true, 
            linkcolor = black, 
            citecolor = black]{hyperref} % For hyperlinks in the PDF
\usepackage{graphicx, multicol} % Enhanced support for graphics
\usepackage{xcolor} % Driver-independent color extensions
\usepackage{marvosym, wasysym} % More symbols
\usepackage{rotating} % Rotation tools
\usepackage{censor} % Facilities for controlling restricted text
\usepackage{listings, style/lstlisting} % Environment for non-formatted code, !uses style file!
\usepackage{pseudocode} % Environment for specifying algorithms in a natural way
\usepackage{style/avm} % Environment for f-structures, !uses style file!
\usepackage{booktabs} % Enhances quality of tables
\usepackage{tikz-qtree} % Easy tree drawing tool
\tikzset{every tree node/.style={align=center,anchor=north},
         level distance=2cm} % Configuration for q-trees
\usepackage{style/btree} % Configuration for b-trees and b+-trees, !uses style file!
\usepackage[backend=biber,style=numeric,
            sorting=nyt]{biblatex} % Complete reimplementation of bibliographic facilities
\addbibresource{ecl.bib}
\usepackage{csquotes} % Context sensitive quotation facilities
\usepackage[yyyymmdd]{datetime} % Uses YEAR-MONTH-DAY format for dates
\renewcommand{\dateseparator}{-} % Sets dateseparator to '-'
\usepackage{fancyhdr} % Headers and footers
\pagestyle{fancy} % All pages have headers and footers
\fancyhead{}\renewcommand{\headrulewidth}{0pt} % Blank out the default header
\fancyfoot[L]{} % Custom footer text
\fancyfoot[C]{} % Custom footer text
\fancyfoot[R]{\thepage} % Custom footer text
\newcommand{\note}[1]{\marginpar{\scriptsize \textcolor{red}{#1}}} % Enables comments in red on margin

%----------------------------------------------------------------------------------------

\begin{document}

%-------------------------------
%	TITLE SECTION
%-------------------------------

\fancyhead[C]{}
\hrule \medskip % Upper rule
\begin{minipage}{0.295\textwidth} 
\raggedright
\footnotesize
ankit kumar jhariya\hfill\\   
ajhariya232@gmail.com\hfill\\
\end{minipage}
\begin{minipage}{0.4\textwidth} 
\centering 
\large 
ASSIGNMENT 4\\ 
\normalsize 
ARTIFICIAL INTELLIGENCE\\ 
\end{minipage}
\begin{minipage}{0.295\textwidth} 
\raggedleft
\today\hfill\\
\end{minipage}
\medskip\hrule 
\bigskip

%-------------------------------
%	CONTENTS
%-------------------------------

\section{Can Artificial Intelligence Demonstrate the paranormal activities?}

\subsection{ The fear of ghosts frightens and fascinates many people.  They fear ghosts, but little evidence exists apart from stories and eerie feelings to show that spirits exist.  Some investigators try to use scientific means to prove the existence of ghosts using photography, electromagnetic detectors, and sound recording; however, the interpretation of signs of ghostly presence often do not stand up to scrutiny.  One source of ghost evidence popular with paranormal investigators goes by the name electronic voice phenomena or EVP.  EVP represents the identification of words or phrases found in electronic recordings, often of static or ambient sounds.  The words or phrases often get attributed to spirits heard in the fuzzy static of a television tuned to an empty channel or the scratchy space between songs on a record.  Because people have active imaginations, they may infer a voice in static that may not be there.  }


\subsection{Some experiences}
Ghost stories both thrill and frighten people all around the world. Spirits appear in literature throughout the ages from the Headless Horseman in the “Legend of Sleepy Hollow to ghosts in Shakespeare’s Richard III and Hamlet. Although there are stories of ghosts abound and many of us have personally experienced the creepy feeling of a paranormal presence, scientific evidence has failed to prove the presence of spirits.  However, artificial intelligence may offer an unbiased approach to detecting paranormal activity. DeepWhisper may not have produced credible evidence of EVP, but remember that voice recognition still needs improvement.

\bigskip

%------------------------------------------------

\section{pseudoscience to science}

\subsection{In experiments conducted by Sparks and collaborators in 1994, researchers manipulated exposure to an episode of a television series about paranormal investigations (Sparks et al., 1994). They found that exposure to one particular episode led participants to express greater belief in paranormal phenomena. On the other hand, exposure to a version of it that included a disclaimer reduced belief in such phenomena. Paul Brewer found similar results in his work in 2012; he examined beliefs about paranormal phenomena such as ghosts and haunted houses and the influence that media messaging about paranormal investigations had on perceptions of how scientific and credible such investigators were. His experiment tested the effects of three different versions of a news story about paranormal investigators on the public. One version presented the news story in terms of traditional supernaturalism, another presented the story with the “trappings of science” including pseudoscientific technology and jargon, a third, discredited the story with the use of a scientific critique.}

\bigskip

%------------------------------------------------

\end{document}
